\documentclass[12pt]{article}
\usepackage[margin=1in]{geometry}
\usepackage{graphics}
\usepackage{fontspec}
\setmainfont[Path=/usr/share/fonts/truetype/calibri/,
    BoldItalicFont=CALIBRIZ.ttf,
    BoldFont      =CALIBRIB.ttf,
    ItalicFont    =CALIBRII.ttf]{Calibri.ttf}

\title{Lab 06: Conservation of Linear Momentum}
\author{Adris Jautakas, partnered with Matthew Kendall}
\date{January 23, 2018}
\begin{document}
   \maketitle

    \pagebreak

    \section*{Abstract}
        \begin{quote}
        {\textit {\small 
            This experiment explores the conservation of linear momentum through
            carts and air tracks. The experiment simulated a collision between
            two low-friction bodies in attempts to verify that their momentum
            is conserved throughout their collision. Despite a considerable
            error, it was clear that the conservation of linear momentum
            holds up in real life.
        } }
        \end{quote}

    \section{Introduction}
        In an ideal physical setting, the linear momentum of any system is conserved.
        This isn't always readily apparent in the real world due to friction, but
        it can be observed in a low friction environment. This experiment involves
        setting up two cars on an air track and pushing them on a collision trajectory
        to observe their behavior as they collide and transfer momentum. We measure
        the effect that an impact has on each object's momentum to confirm that it
        is conserved in the system.
    \section{Materials}
        \begin{itemize}
            \item Air Track
            \item Three spring loaded Carts: Two of similar mass
            \item Two Artificial barriers / walls attachable to table
            \item Four Timers
            \item Meterstick
            \item Scale
        \end{itemize}
    
    \section{Methods}
        \subsection{Experiment Setup}
            Place the track flat on a table. Attach artificial barriers/walls to
            the table such that they are perpendicular to and at both end of the 
            track. Then place a spring loaded cart against the wall with the spring
            pointing at the wall.
        \subsection{Experimental Setup Trial}
            For the trial run, no extra setup is needed.
        \subsection{Experimental Setup A: Stationary cart}
            Place a second cart in the middle of the track at a recorded distance.
        \subsection{Experimental Setup B: Equal and opposite motion}
            Instead of placing the second cart in the middle of the track, place
            it against the other side of the track, spring pointed at the other
            wall.

        \subsection{Procedure: Trial and measurements}
            \par First find the mass of all carts with the scale.
            \par After setting up the experiment, a trial run will be conducted to
            determine the method for measuring change in time. As the carts move,
            every experiment should measure the time it takes for them to travel
            a certain distance, and the method for doing this varies. One method
            involves four observers, each measuring the time it takes for a cart
            to cross an individual point from the start, and the other method
            has two observers measure the duration it takes a cart to travel
            between two points. Both measurement methods are measuring the same
            thing, but their accuracy may differ.
            \par For the trial run, a cart's velocity must be detected using both
            methods of experiment. As the cart passes through a set of defined
            points, the time it takes for the cart to pass through each point is
            determined. After running the trials, determine which method results
            in the most consistent and accurate reading for velocity and use
            that method for the remainder of the experiments.
        \subsection{Procedure A: Stationary Cart}
            \par After completing Experimental Setup A, launch the cart from the 
            wall and read the time measurements. These measurements should tell
            the speed of both carts before and after their collision, and they
            should be read before and after their collisions (the stationary
            cart is assumed to have 0 velocity before collision). This is tested
            both with two carts of equal mass and carts of inequal mass.
        \subsection{Procedure B: Equal and opposite motion}
            \par After completing Experimental Setup B, launch both carts at the
            same time, opposite to each other. Record their times both before
            and after their collisions to grab their veolcities before and after
            their collisions.

        \section{Data}

            % Trial data table
            {\large Trial run}
            \begin{center}
                \begin{tabular} {|c|c|c|}
                \hline
                     & Method A & Method B \\
                \hline
                    \begin{tabular} {c}
                        car mass (g) \\
                    \hline
                        \begin{tabular} {c|c}
                            $m_1$ & 680 \\
                        \hline
                            $m_2$ & 685 \\
                        \hline
                        \end{tabular}
                    \end{tabular}
                &
                    \begin{tabular} {c|c|c|c|c|c|c}
                        $t_1$ (s) & $t_2$ (s) & $\Delta t_{12}$ (s) & $t_3$ (s) & $t_4$ (s) & $\Delta t_{34}$ & $\%$ Diff \\                \hline
                        1.3 & 2.8 & 1.5 & 1.7 & 3.5 & 1.8 & $9\%$ \\
                    \hline
                        1.3 & 2.74 & 1.44 & 1.6 & 3.2 & 1.6 & $5.3\%$ \\
                    \hline
                    \end{tabular}
                &
                    \begin{tabular} {c|c|c}
                        $\Delta t_{12}$ & $\Delta t_{34}$ & $\%$ Diff \\
                    \hline
                        0.21 & 0.15 & $7.60\%$ \\
                    \hline
                        0.22 & 0.27 & $7.20\%$ \\
                    \hline
                    \end{tabular} \\
                    \hline
                \end{tabular}
            \end{center}

            % Table 1
            {\large Experiment 1: $m_1 = m_2$ and $v_{2_i} = 0$}
            \begin{center}
                \begin{tabular} {|c|c|c|}
                    \hline
                    \begin{tabular}{c}
                        Trial \\ \hline 1 \\ \hline 2 \\ \hline 3 \\ \hline
                    \end{tabular}
                &
                    \begin{tabular}{c}
                        Before Collision \\ \hline $m_1$ \\ \hline
                        \begin{tabular}{c|c|c}
                            $\Delta t_1$ & $v_{1_i}$ & $p_{1_i}$ \\
                        \hline
                            3.10 & 0.32 & 0.22 \\
                        \hline
                            2.70 & 0.37 & 0.25 \\
                        \hline
                            3.40 & 0.29 & 0.20 \\
                        \end{tabular}
                    \end{tabular}
                &
                    \begin{tabular}{c}
                        After Collision \\ \hline $m_2$ \\ \hline
                        \begin{tabular}{c|c|c|c}
                            $\Delta t_2$ & $v_{2_f}$ & $p_{2_f}$ & $\%$ Diff \\
                        \hline
                            3.40 & 0.29 & 0.20 & $4.80\%$ \\
                        \hline
                            3.30 & 0.30 & 0.21 & $8.7\%$ \\
                        \hline
                            3.70 & 0.27 & 0.19 & $2.60\%$ \\
                        \end{tabular}
                    \end{tabular} \\
                    \hline
                \end{tabular}
            \end{center}

            % Table 2
            {\large Experiment 2: $m_3 > m_1$ and $v_{3_i} = 0$}
            \begin{center}
                \begin{tabular} {|c|c|c|}
                    \hline
                    \begin{tabular}{c}
                        Trial \\ \hline 1 \\ \hline 2 \\ \hline 3 \\ \hline
                    \end{tabular}
                &
                    \begin{tabular}{c}
                        Before Collision \\ \hline $m_1$ \\ \hline
                        \begin{tabular}{c|c|c}
                            $\Delta t_1$ & $v_{1_i}$ & Total $p$ \\
                        \hline
                            3.20 & 0.31 & 0.21 \\
                        \hline
                            3.10 & 0.32 & 0.22 \\
                        \hline
                            2.70 & 0.37 & 0.25 \\
                        \end{tabular}
                    \end{tabular}
                &
                    \begin{tabular}{c}
                        After Collision \\ \hline $m_2$ \\ \hline
                        \begin{tabular}{c|c|c|c|c|c}
                            $\Delta t_{1_f}$ & $v_{1_f}$ & $p_{1_f}$ & $\Delta t_{3_f}$ & $v_{3_f}$ & $p_{3_f}$ \\
                        \hline
                            1.30 & 0.77 & 0.20 & 2.20 & 0.61 & 0.54 \\
                        \hline
                            1.60 & 0.63 & 0.18 & 1.90 & 0.92 & 0.81 \\
                        \hline
                            2.20 & 0.45 & 0.24 & 2.10 & 0.76 & 0.71 \\
                        \end{tabular}
                    \end{tabular} \\
                    \hline
                \end{tabular}
            \end{center}

            % Table 3
            {\large Experiment 3: $m_1 = m_2$ and $v$ is opposite}
            \\
            BEFORE collision
            \begin{center}
                \begin{tabular} {|c|c|c|}
                    \hline
                    \begin{tabular}{c}
                        Trial \\ \hline 1 \\ \hline 2 \\ \hline 3 \\ \hline
                    \end{tabular}
                &
                    \begin{tabular}{c}
                        $m_1$ \\ \hline
                        \begin{tabular}{c|c|c}
                            $\Delta t_1$ & $v_{1_i}$ & $p_{1_i}$ \\
                        \hline
                            2,80 & 0.36 & 0.24 \\
                        \hline
                            2.9 & 0.34 & 0.23 \\
                        \hline
                            3.20 & 0.31 & 0.21 \\
                        \end{tabular}
                    \end{tabular}
                &
                    \begin{tabular}{c}
                        $m_2$ \\ \hline
                        \begin{tabular}{c|c|c}
                            $\Delta t_2$ & $v_{2_f}$ & $p_{2_f}$ \\
                        \hline
                            4.20 & 0.24 & 0.31 \\
                        \hline
                            3.70 & 0.27 & 0.27 \\
                        \hline
                            2.90 & 0.34 & 0.34 \\
                        \end{tabular}
                    \end{tabular} \\
                    \hline
                \end{tabular}
            \end{center}
            
            AFTER collision
            \begin{center}
                \begin{tabular} {|c|c|c|}
                    \hline
                    \begin{tabular}{c}
                        Trial \\ \hline 1 \\ \hline 2 \\ \hline 3 \\ \hline
                    \end{tabular}
                &
                    \begin{tabular}{c}
                        $m_1$ \\ \hline
                        \begin{tabular}{c|c|c}
                            $\Delta t_1$ & $v_{1_i}$ & $p_{1_i}$ \\
                        \hline
                            2.5 & 0.4 & 0.21 \\
                        \hline
                            2.3 & 0.43 & 0.26 \\
                        \hline
                            3.1 & 0.42 & 0.33 \\
                        \end{tabular}
                    \end{tabular}
                &
                    \begin{tabular}{c}
                        $m_2$ \\ \hline
                        \begin{tabular}{c|c|c}
                            $\Delta t_2$ & $v_{2_f}$ & $p_{2_f}$ \\
                        \hline
                            2.5 & 0.4 & 0.34 \\
                        \hline
                            2.7 & 0.37 & 0.37 \\
                        \hline
                            3.4 & 0.29 & 0.43 \\
                        \end{tabular}
                    \end{tabular} \\
                    \hline
                \end{tabular}
            \end{center}

    \section{Analysis}
        \subsection{Equations}
            Linear momentum is described mathematically as $p = mv$ where $m$
            is an object's mass and $v$ is its velocity. The conservation
            of momentum states that the net momentum in a system remains
            constant, which means it doesn't change, so $\Delta p = o$. In
            the case of the masses, the net $p$ can be described as
            $m_1 v_1 + m_2 v_2$ which arrives at the equation $m_{1_i} +
            m_{2_i} = m_{1_f} + m_{2_f}$ where $i$ subscript means initial
            and $f$ subscript means final. If the system conserves momentum,
            the sum of its momentums will remain constant
        \subsection{Evaluation of Data}
            \par Upon running the trial, the mean percent errors were very similar
            for both trials. However, the error in Method B (where there were 
            two observers measuring time between two points) was a lot more
            centralized and less fluctuating, making it slightly more reliable,
            which made it our method of choice.
            \par For each experiment, the total momentum should be constant. This
            means that $p_{1_i}+ p_{2_i}$ should equal $p_{1_f} + p_{2_f}$.
            \par The percent error in this experiment can be due to a variety of
            factors, most notably the inelasticity of the carts as they collide
            and the friction between the track and its wheels that adds up. These
            may have tampered and decreased the energy of the system, affecting
            our measurements.
    \section{Conclusion}
        Overall, it seems like this experiment can verify the conservation of momentum
        is in fact a fesible reality. Throughout our experiments, our largest percent
        difference between our initial and final total momentum was 16\%, which is
        considerable but it does not completely refute our claim.
        
\end{document}
