\documentclass[12pt]{article}
\usepackage[margin=1in]{geometry}
\title{Lab 08: Hooke's Law and Simple Harmonic Motion}
\author{Adris Jautakas, partnered with Matthew Kendall}
\date{January 14, 2018}
\begin{document}
   \maketitle

    \section*{Abstract}
        \begin{quote}
        {\textit {\small 
            Abstract goes here
        } }
        \end{quote}

    \section{Introduction}
        Hooke's law and harmonic motion describe the behavior of a spring, both
        in the force that a spring exerts and in its motion. This lab attempts
        to verify these equations by measuring the elongation of a spring and
        the period of an oscillating spring, adjusting the mass of the spring's
        weights. After performing experiments and measurements, we check
        our results by plugging them back into the equation.
        
    \section{Materials}
        \begin{itemize}
            \item Rubber Band
            \item Pole with horizontal bar
            \item Spring
            \item Weights (of mass 5g, 10g, 20g, 50g, and 100g)
            \item Meter stick
            \item Clips (to attach pole, horizontal bar, and meter stick)
            \item Stopwatch
        \end{itemize}
    
    \section{Methods}
        \subsection{Experiment Setup}
            The pole must first be fixated perpendicular to the ground.
            Then, a horizontal bar should be attached to the pole such that
            masses can be hung from the bar without the bar moving or falling
            off. Then there are three experiments run on this setup
        \subsection{Experiment A: Rubber Band}
            uh
    \section{Data}
        Data
    \section{Analysis}
        \subsection{Equations}
        \subsection{Evaluation of Data}
        Analysis 
    \section{Conclusion}
        Conclusion
\end{document}
