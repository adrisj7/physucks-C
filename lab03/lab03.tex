\documentclass[12pt]{article}
\usepackage[margin=1in]{geometry}
\title{Lab 03: Newton's Second Law: The Atwood Machine}
\author{Adris Jautakas, partnered with Matthew Kendall}
\date{November 30, 2017}
\begin{document}
   \maketitle

    \pagebreak

    \section*{Abstract}
        \begin{quote}
        {\textit {\small 
            Newton's Second law of motion, $F = ma$, describes the relationship
            between Force, Mass, and Acceleration when a force acts on an
            object. This lab attempts to experimentally verify this law with
            an Atwood machine, which uses pulleys to distribute and adjust 
            the opposing gravitational forces of two weights. This setup
            gives us the mass, acceleration, and force of a moving body,
            which we can compare to test for Newton's Second Law. Our
            findings have shown that Newton's Second Law is in fact
            accurate, despite errors in the experimental results.
        } }
        \end{quote}
    \section{Introduction}
        \par Newton's Laws and Newtonian physics accurately describe the 
        relationships between point particle motion and the forces
        that act on these particles.
        Newton's Second Law describes the mathematical relationship 
        between a force acting on a body, the body's mass, and the
        acceleration that the body experiences. This is expressed with
        the equation $F_{net} = ma$ where $F_{net}$ is the net force experienced
        by an object, $m$ is its mass and $a$ is the acceleration the object
        experiences
        \par Newton's Second Law can be tested experimentally through a variety
        of physical methods. The following experiment uses an Atwood Machine,
        a device that consists of two weights connected through a string, run
        through an eleveated pulley. By finding the apparent force of gravity
        and the acceleration of the masses, one can test Newton's Second Law
        by plugging measurements into $F_{net} = ma$.

        %The masses are both suspended, hanging on
        %the pulley, and gravity appears to exer a pulling force on both masses.
        %These apparent forces result in a net acceleration that depends on the
        %mass of the weights. By knowing the force that gravity exerts on the
        %masses, we can

    \section{Materials}
        \begin{itemize}
            \item Erectable vertical bar
            \item Pulley, attachable to the vertical bar (with clamps)
            \item Weights (of mass 5g, 10g, 20g, 50g, and 100g)
            \item String, attachable to weights
            \item Paper clips
            \item Meter Stick
            \item Stopwatch
            \item Scale (electronic or triple beam balance)
        \end{itemize}
    \section{Methods}
        \subsection{Experiment Setup}
            Fixate the vertical bar perpendicular to the ground. Attach the 
            pulley to the vertical bar and run the string over the pulley.
            Tie a weight to each side of the string such that both weights are
            suspended. If the weights have equal mass, they should hang in the
            air. If one side has more mass, the string will move towards the
            heavier mass, and the heavier mass will fall down while the lighter
            accelerates upwards. 
        \subsection{Accounting for Friction}
            \par Experiments always have some kind of natural error, and this one
            is no exception. The force of friction between the pulleys is
            strong enough to have a measurable impact on our data.
            However, it can be experimentally negated to
            reduce the effect of friction on our measurements.
            \par This is done by placing two weights of equal mass on both sides
            of the pulley, and adding tiny masses (paper clips) to one side until
            a short tug on one side makes the masses move at a constant velocity.
            At this point, the force of gravity equals the force of the pulley's 
            kinetic friction, and the effect of friction on the system is
            negated

        \subsection{Experiment 1: Constant Force}
            After friction has been negated, the machine is used for a series of
            tests. Each test start with an imbalance of the attwood machine masses,
            with the heavier mass raised at a recorded height, which varies due to
            the string stretching with heavier masses. The heavier mass is
            then dropped, and the time between the drop and the weight's collision.
            This experiment begins with two offset masses, and for each trial
            both masses have extra weight added equally to both sides at 100g
            intervals. This ensures that the net force of gravity is constant,
            even when the masses change. This means that the force of friction 
            from the pulley varies (due to the weight's higher normal force)
            and it must be redetermined for every trial
        \subsection{Experiment 2: Constant Mass}
            These experiments are repeated again, but instead of keeping the force
            of gravity constant, the net mass of the system is kept constant and
            what changes is the inbalance of the system, by shifting weight from 
            one mass to another. This means that the net mass is constant,
            and the force of friction does not have to be redetermined for
            every trial.
    \section{Data}
        {\large Experiment 1: Constant Force}
        \begin{center}
    
            \begin{tabular} {|c|c|c|c|c|}
            \hline
                Descending Mass $m_2$ (g) & 60 & 160 & 260 & 360 \\
                Descending Mass $m_1$ (g) & 50 & 150 & 250 & 350 \\
            \hline
            \hline
                Distance of travel $y$ (cm) & 57.5 & 57.5 & 55.0 & 54.0 \\
            \hline
            \hline
                Trial 1 (s) & 1.02 & 2.09 & 2.76 & 3.53 \\
                Trial 2 (s) & 1.26 & 2.04 & 2.71 & 3.55 \\
                Trial 3 (s) & 1.06 & 2.00 & 2.75 & 3.56 \\
                Average Time (s) & 1.11 & 2.04 & 2.74 & 3.55 \\
            \hline
                
            \end{tabular}
        \end{center}

        {\large Experiment 2: Constant Mass}
        \begin{center}

            \begin{tabular} {|c c|}
                \hline
                Mass of frition weight $M_f$ & = 1.3 g \\
                Total mass of system: $M_{net}$ & = 500.3 g \\
                \hline
            \end{tabular}

            \begin{tabular} {|c|c|c|c|c|}
            \hline %FIXME
                Descending Mass $m_2$ (g) & 250.3 &  &  &  \\
                Descending Mass $m_1$ (g) & 250 &  &  &  \\
            \hline
            \hline
                Distance of travel $y$ (cm) &  &  &  &  \\
            \hline
            \hline
                Trial 1 (s) & 2.66 & 1.73 & 1.39 & 1.02 \\
                Trial 2 (s) & 2.55 & 1.74 & 1.34 & 1.15 \\
                Trial 3 (s) & 2.55 & 1.73 & 1.39 & 1.02 \\
                Average Time (s) & 2.59 & 1.73 & 1.36 & 1.06 \\
            \hline
                
            \end{tabular}
        \end{center}
    \section{Analysis}
        \subsection{Equations}
            \par The force of gravity exerted on any object is directly dependent on 
            its mass. This follows the equation
            \begin{equation}
                F_g = mg
            \end{equation}
            where $F_g$ is the force of gravity, $m$ is the object mass and $g$ is
            the acceleration due to gravity ($9.81 m/s^2$ on Earth) 
            \par The two masses in the pulley system create a set of opposing 
            gravitational forces. Because these forces directly oppose each other,
            the net force of the system can be defined as such:
            \begin{equation}
                F_{net} = F_{g1} - F_{g2} = m_1g - m_2g
            \end{equation}
            Where $F_{net}$ is the net force, $m_1$ and $m_2$ are the masses of each 
            weight, and $g$ is acceleration due to gravity.
            \par This mathematical description is possible because the two masses
            and their forces due to gravity behave like a one dimensional system 
            because they are parallel to each other.
            \par We can test Newton's Second law, $F = ma$, with the information
            provided. We have three masses in play: the two weights, and a friction
            counterweight. We know the net force of the system, and the acceleration
            of the system. If we use newton's law to predict the acceleration of the
            system, we can verify whether the law is accurate.
            \begin{equation}
                a = \frac{F_{net}}{m_{total}} = \frac{m_1g - m_2g - m_fg}{m_1 + m_2 + m_f}
            \end{equation}
            Both experiments use the same equations to predict and test Newton's 
            Second Law.
            \par To determine the acceleration of the body, we use the time and 
            height that the masses fall from with kinematic equations for constant
            acceleration. \\
            We know that
            \begin{equation}
                h = \frac{1}{2}at^2
            \end{equation}
            Where $d$ is the starting height of the mass, $a$ is its acceleration,
            and $t$ is the time elapsed as the mass travels from its height to
            the ground at $h = 0$ \\
            \par We can rewrite the equation as
            \begin{equation}
                a = \frac{2h}{t^2}
            \end{equation}
            to find the acceleration from our recorded data.
        \subsection{Evaluation of Data}
           Our percent error for experiment 1 falls within 10\%,
           while our percent error for experiment 2 falls within 13\%.
           This shows that our results are relatively accurate
    \section{Conclusion}
        Overall, through a shockingly unexpected series of events,
        Newton's Second Law does in fact appear to be correct due to
        our consistent results and small experimental error. \\
        The error in this experiment could be attributed to a variety of
        factors. This includes the weakening and lengthening of the rope 
        over time due to prolonged stress, and the limited accuracy
        that paperclips provide as a friction mass.
        
\end{document}
